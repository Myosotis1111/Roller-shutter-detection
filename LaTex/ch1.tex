\chapter{Introduction}\chaplabel{1}

Building management is one key aspect of facility management. It integrates various disciplines to ensure the functionality, comfort, safety, and efficiency of the building environment for people, processes, and technology. Energy management has become the major aspect of building management nowadays. The calculation of energy consumption of a building is heavily influenced by windows and their roller shutters: while windows emit most of the heat generated by sunlight to the environment, the roller shutters may have a huge impact on such transfer process. In order to optimize the energy consumption of the building, efficient automated building management and monitoring should incorporate available information of the building status. This includes the window status and the roller shutter positions.

 This bachelor thesis aims to investigate and implement in machine vision algorithm for detection of the roller shutter status. Since the estimation of energy consumption is a continuous process, the implemented algorithm should be capable of accurate detection 24/7 and under various weather conditions. In this bachelor thesis, deep learning algorithm You only look once (YOLO) is selected to solve this object detection problem. The detection model is trained and tested via a limited data set obtained manually or by preset cameras on the THL campus.

 This bachelor thesis is organized into three parts. The first part (Chapter 2-4) is a literature review. In Chapter 2, several common solutions are systematically introduced, which includes computer vision, machine learning and Convolutional Neural Networks (CNNs). Chapter 3 explains the applications of CNNs on object detection and main features of YOLO algorithm. Chapter 4 analyzes the existing challenges in roller shutter detection under the requirements of this dissertation.

